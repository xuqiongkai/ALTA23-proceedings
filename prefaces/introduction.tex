Welcome to ALTA2023, the 21st Annual Workshop of the Australasian Language Technology Association. We're thrilled to have you join us in Melbourne, Australia, from November 29 to December 1, 2023. After several years of hybrid format due to COVID, this year ALTA will largely operate as an offline event with limited online capability, so as to encourage the community to come together in person.\newline

ALTA is the premier workshop for natural language processing or computational linguistics in the Australasia region (despite being downgraded to C by CORE2023) and it is now indexed by Scopus (after efforts by the ALTA executive committee). This year the programme includes 3 keynote talks (with 1 more joint keynote with AI@Melbourne Connect Symposium), 1 tutorial, 1 panel discussion, 5 oral sessions and 1 poster/demo session. In addition to long/short academic papers, shared task papers and non-archival abstract presentations, we've introduced a new category of submission --- industry demonstrations --- to foster greater industry involvement. In terms of submission statistics, we received 25 long/short papers, 6 shared task papers, 6 abstract presentations and 2 industry demonstrations, and accepted 17 long/short papers (breakdown: 10 long and 7 short) and all submitted shared task papers, abstract presentations and industry demonstrations. Our acceptance rate for long/short papers (68\%) aligns with last year's figures. Note that all long/short papers went through the double-blind peer-review process (shared task papers, abstract presentations and industry demonstrations, however, did not). \newline

2023 has been an interesting year. We saw large language models breaking into mainstream consciousness, and within a year (ChatGPT was released on 30 November 2022) they have transformed the field in both academia and industry. This is reflected in our keynote talks, panel discussion and industry demonstrations, which all feature large language models to some extent. One of the accepted papers even credits a large language model as a co-author.\newline

This workshop could not have happened without the help and enthusiastic participation of many parties, and we would like to give a big `thank you' to all of them. Specifically, we want to thank our keynote speakers --- Reza Haffari (Monash), Heng Ji (Illinois Urbana-Champaign) and Terrence Szymanski (SEEK) --- for their inspiring talks. Special thanks to the organising and program committee whose hard work made ALTA a reality. Lastly, we want to express our appreciation to our sponsors: Melbourne Connect and Telstra (Platinum); The University of Melbourne AI Group, Google and Defense Science and Technology Group (Gold); SEEK (Silver); and Redenlab and Commonwealth Bank (Bronze). 2023 financially hasn't been the best year, and we're incredibly thankful for the support you've provided.\newline

Welcome to Melbourne! Our submissions have come from many places, and we look forward to a rich and rewarding time together.\newline\newline\newline

Jey Han Lau\newline

Program Chair
